\section{Methods and materials}
\begin{itemize}
\item 1 six-year old grapefruit plant
\item 1 cord of known spring constant
\item 1 measuring tape
\item 1 Kaz Inc. Ht-908 15 inch Honeywell Turbo Force Room Air Circulator Fan
\item colored construction paper
\item 1 pair of scissors
\item clear tape
\item 1 Samsung Galaxy S-8 smartphone camera
\item 1 wooden chair
\item flat, rectangular surfaces
\item MATLAB software
\item pipe cleaners
\item nylon rope
\end{itemize}

The first step in the process was to find a cord of known spring constant. In this experiment, a cord from a party mask was found to have a spring coefficient of 0.589 lbf/in. The cord was attached to a nylon rope and tied to pipe cleaners at both ends so that it could be connected to plant stems. The pipe cleaners were included in the spring constant calculation.

The experimental setup for measuring the spring's deflection due to drag is shown in \fref{fig:methods1}. The cord was strung between a heavy chair and the trunk of a grapefruit plant so that it was not slack, and its length was measured using a tape measure. Then, a box fan was placed one foot away from the trunk of the plant. The plant and fan were raised using flat, rectangular objects, such as textbooks, in order to position the plant's canopy in front of the fan. The cord's deflection was measured five times at each of the fan's three speed settings and a slow motion video of the plant's leaves was taken at the highest fan setting.

Next, the cross sectional area of the plant was calculated using color blob detection in Matlab. To do this, black construction paper was cut to the size and shape of the fan's opening and affixed to it. Because air was blowing from relatively close to the plant, only the area which was directly in front of the fan opening would be exposed to wind. The plant was of a complex shape, so it was more efficient to calculate the area of the fan which was not covered by the plant. The setup was photographed from afar and scaled to eliminate the colorful surroundings, as shown in \fref{fig:methods2}. A picture of the fan-shaped paper was also taken against a blue background from the same distance with no citrus obscuring it and scaled by the same amount, which can also be seen in \fref{fig:methods2}. Using Matlab's colorThresholder system, the ratio of covered to uncovered area was found and multiplied by the known paper area to find the area of the grapefruit leaves.

After analyzing the results of the grapefruit plant, a rigid model was created to test the same conditions without the presence of bending. A small leaf was placed against an aluminum Chipotle bowl lid to serve as a stencil, which was cut out using scissors. Then, this model leaf was taped to a wooden pencil so that the pencil would prevent it from flexing, while any tape edges were cut off. The finished leaf model can be seen in 
\fref{fig:methods3}.

The leaf model was tested in the same manner as the grapefruit plant at the highest fan setting. Because the deflection in the cord was quite small, the displacement was measured using slow motion video. To do this, a tape measure with millimeters marked was taped below the cord and the video was taken from directly above. The video device was planted on a solid surface to minimize movement, and the displacement was found by zooming in on the video and pausing when the cord was fully extended and relaxed. 

In order to compare the model area to the area of the grapefruit, a sheet of orange construction paper was photographed with and without the model leaf against it. Then, using the color blob detection technique in Matlab, the ratio of the model area to the paper area was calculated and multiplied by the actual paper area, in square feet. The images for this analysis can be seen in \fref{fig:methods4}.

With the approximate cross sectional areas of the actual grapefruit leaves and the model found, the discussion over how these areas and their corresponding drags differed could begin.


% Fig 1 is good but cluttered, can we zoom in on only the business part; or turn it into a drawing. Altnatively, add callouts and scale bar? Is there a shot more from the side, that shows the experimental rig without foreshortening? 
\begin{figure}
\begin{center}
\includegraphics[width=0.5\columnwidth]{figures/Grapefruit_Setup.jpg} 
\end{center}
\caption{Experimental setup to measure the deflection of a grapefruit plant when exposed to wind.}
\label{fig:methods1}
\end{figure}

% Not sure what you get from Fig 2; I might move it to appendix. 
\begin{figure}
\begin{center}
\includegraphics[width=0.33\columnwidth]{figures/Fan.jpg}
\includegraphics[width=0.33\columnwidth]{figures/Fan1.jpg}
\end{center}
\caption{The exposed area of the fan-shaped construction paper with and without the plant in front of it.}
\label{fig:methods2}
\end{figure}

% Fig 3 is good but zoom in on the two, add callouts and a scalebar, maybe rotate leafs into their normal positions. 
\begin{figure}
\begin{center}
\includegraphics[width=0.5\columnwidth]{figures/Metal_Leaf_Compared.jpg}
\end{center}
\caption{The aluminum grapefruit leaf model (right) with a small jackfruit leaf, for scale.}
\label{fig:methods3}
\end{figure}

% Not sure what Fig 4 adds, might move to appendix. 
\begin{figure}
\begin{center}
\includegraphics[width=0.33\columnwidth]{figures/MetalLeaf.jpg}
\includegraphics[width=0.33\columnwidth]{figures/Paper.jpg}
\end{center}
\caption{The orange construction paper of known area photographed with and without the model leaf overlaid.}
\label{fig:methods4}
\end{figure}

\subsection{Specimens and physical models}
I used a single \SI{6}{y} old specimen of \emph{Citrus x paradisi} (grapefruit) for all measurements. (words about how tall the specimen was and what conditions it was raised in, since plants are plastic). 

To remove the effects of flexibility, I also created a physical model of a single \emph{C. x paradisi} leaf using (x thick) aluminum sheeting from a food container (Chipotle; (location)). To prepare the physical model, I traced an actual leaf and cut the profile of the model to match. The physical model was mounted on a wooden pencil to provide a rigid attachment point compared to the typical flexible leaf petioles on \emph{C. x paradisi}. 

\subsection{Drag measurements}
(fill in this from other stuff and move other stuff to Appendix)


%\subsection{Statistical analyses}
Statistical analyses of the effects of both leaf and fan speed on drag and drag/area were performed using R \citep{r2020} using two-way analysis of variance (ANOVA); plots were prepared using the \lstinline{tidyverse} and \lstinline{ggplot2} libraries \citep{wickham2019tidyverse}. The average drag per unit area was found for the model leaf and the grapefruit leaves.

