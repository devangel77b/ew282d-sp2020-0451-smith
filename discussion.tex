\section{Discussion}
\label{sec:discussion}

\subsection{Leaf flexibility reduces drag relative to a rigid physical model}
Drag/area ratio is 19\% higher in the rigid physical model compared to the live \Cxparadisi\ specimen (\fref{fig:results:dragarea}; \fref{tab:results:drag}). While the small number of replicates and measurement noise limit the strength of my conclusions, the result suggests leaf flexibility is able to reduce drag experienced by the whole plant. 

Slow motion video frames (\fref{fig:results:leafmovement}) suggest a mechanism: bending at the leaf petiole. As \fref{fig:results:leafmovement} shows, the leaves were able to bend quite far at their joints before further deforming to reduce drag. They also avoided rapid flapping through the rigidity offered by the petioles. It would appear that the petioles gave the leaves a pivot point which allowed them to reposition effectively, but the petioles' firm, single degree-of-freedom, connections to the stem kept the leaves from experiencing vortex-induced rotational modes or the complex fluttering observed by \citet{miller2012reconfiguration}.

\subsection{Where is flexibility most useful?}
My observations suggest distal bending at, at the petioles and nearest the leaves, allow for protective load shedding of wind loads in \Cxparadisi. The drag measurements were taken from a point halfway down the trunk of the plant and the model, and therefore did not capture the torque experienced by the petiole joints. Judging from the immense range of motion in all three axes shown in the slow motion videos, it is reasonable to assume that the leaves' movements produced disproportionately large torque on their petiole joints, while producing less drag on the entire plant when compared to a rigid model. 

The petiole joints are more likely to give under stress, before the plant itself is uprooted or a branch detached. Anecdotally, as the plant's caretaker, I have observed that high winds in the spring resulted in loss of several large leaves on this specimen. In each case the petiole was consistently the fracture point. The joints thus play a dual function, reducing the plant's overall drag while also reducing the negative effects of fluttering to the leaves themselves rather than the plant as a whole. In the highest winds, the petioles provide a sacrificial mechanism to reduce load and avoid catastrophic loss of the entire plant. 
