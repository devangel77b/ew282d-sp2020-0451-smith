\section{Discussion}
%Here discuss what your results mean. Are your hypotheses supported or not? Do you have an answer to your overall research question?

The drag to area ratio showed the effectiveness of the petiole joints numerically, while the slow-motion video analysis suggests the reason behind the statistically significant results. This relationship implies a disposition to protective load shedding in \Cxparadisi.

The 17\% difference between the model drag to area ratio and that of the actual plant indicate that the leaf petioles did indeed decrease the amount of drag experienced by the plant. As the slow motion video showed, the leaves were able to bend quite far at their joints before further deforming to reduce drag. They also avoided rapid flapping through the rigidity offered by the petioles. It would appear that the petioles gave the leaves a pivot point which allowed them to reposition effectively, but the petioles' firm connections to the stem kept the leaves from experiencing the complex fluttering observed by \citet{miller2012reconfiguration}.

The drag measurements were taken from a point halfway down the trunk of the plant and the model, and therefore did not capture the torque experienced by the petiole joints. Judging from the immense range of motion in all three axes shown in the slow motion videos, it is reasonable to assume that the leaves' movements produced disproportionately large torque on their petiole joints, while producing significantly less drag on the entire plant when compared to a rigid model. This nature implies that the petiole joints are more likely to give under stress before the plant itself is uprooted or a branch detached. While more official studies will need to be made to quantify the torque on the petiole joints, I can say from personal experience, as the plant's caretaker, that high winds in the spring have taken off several large leaves on this specific specimen, with the petiole being their consistent fracture point. The joints thus play a dual function, reducing the plant's overall drag while also reducing the negative effects of fluttering to the leaves themselves, rather than the plant, as a whole.
