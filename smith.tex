\documentclass[]{article}

\title{Drag on grapefruit leaves in wind and flow-induced shape reconfiguration}
\author{C Smith}
\date{\today}


\usepackage[separate-uncertainty=true]{siunitx}
\DeclareSIUnit{\year}{y}
\DeclareSIUnit{\inch}{in}
\DeclareSIUnit{\foot}{ft}
\DeclareSIUnit{\poundforce}{lbf}
\DeclareSIUnit{\pound}{lbm}
\DeclareSIUnit{\frame}{frame}
\usepackage{graphicx}
\usepackage[round,authoryear]{natbib}
\bibliographystyle{apalike}
\usepackage[plain]{fancyref}
\usepackage{listings}
\usepackage{amsmath,amsfonts,amssymb}
\usepackage{booktabs}
\lstset{%
  basicstyle=\ttfamily,
  columns=fullflexible,
  showstringspaces=false}
\usepackage[dvipsnames,svgnames]{xcolor}
\usepackage{hyperref}
\hypersetup{%
  colorlinks=true,
  linkcolor=violet,
  urlcolor=blue,
  citecolor=blue}
\usepackage{svg}
%\usepackage{svg-extract}
\usepackage{fullpage}

\newcommand{\Citrusxparadisi}{\emph{Citrus $\times$ paradisi}}
\newcommand{\Cxparadisi}{\emph{C.~$\times$~paradisi}}
\newcommand{\Matlab}{Matlab}
\begin{document}

\maketitle
\begin{abstract}
This study was conducted to investigate how effectively citrus plants can change their structure to generate less drag in wind. A cord with known spring constant was attached to a young grapefruit plant and an aluminum model of one of its leaves, a box fan was used to create a wind tunnel, and area was calculated using \Matlab's color thresholding and blob detection software. The actual drag created on the grapefruit leaves was significantly less than the drag expected on an identical model, and slow motion video analysis indicated that this was the result of the citrus's ability to bend at the petiole. The extreme bending in several directions showed the plant's possible inclination toward protective load shedding.
\end{abstract}
{\scriptsize\textbf{Keywords: }grapefruit, \Citrusxparadisi, drag, leaf, petiole}

\section{Introduction}
%Smith: The main ones to look at closely are Vogel and also Laura Miller and Company, and then the algae stuff from Emily Carrington, Mark Denny, Hannah Stewart, etc. May also search on ecological safety factors, reconfigurable structures, use of physical models. I didn't check mangrove literature for drag and shape reconfiguration, I think there's also coral literature about corals breaking during hurricanes to reduce loads / save the colony from death. In engineering context, this is same idea as protective load shedding in electrical systems, or crumple zones and holding bulkheads in structural systems; reefing during sailing, etc. 

%Work from an inverted triangle (broader topics to more specific). Explain what you are interested, review some relevant literature, and then set up what your specific research question is. Cite literature using the author-year format, as in \citep{buck2020go}. If you need pictures to explain the relevant biomechanics, feel free to include. This section should also say a little why your research matters. 
% things to explain include drag, how it depends on speed, area, etc; why it matters to plants, what they can do to minimize or mitigate drag...
%The last part of this section should be the specific hypotheses you seek to test. 

Plants possess unique structural designs and material properties which allow them to combat the negative effects of drag in their environments. As explained by \citet{delangre2008effects}, fluid flow over plant canopies creates a drag force on the structure through both pressure difference and friction. In the classic drag equation, both of these factors are combined into the singular drag coefficient, creating the relationship
\begin{equation}
F=0.5 C_D \rho V^2 A
\label{eq:drag}
\end{equation}
where $F$ is the total drag force, $C_D$ is the drag coefficient, $\rho=\SI{1.204}{\kilo\gram\per\meter\cubed}$ is the density of air, $V$ is relative fluid velocity, and $A$ is a reference area, here taken as the cross-sectional area of the surface normal to the flow direction. Plants use their flexible structures to reconfigure under drag load, rolling their leaves into cones in order to minimize cross sectional area, drag coefficient, and the oscillations of vortex shedding \citep{vogel1989drag}. This allows plants to endure the onslaught of heavy winds or floodwaters with much lower risk for structural damage than if they were to be entirely rigid. 

However, there is room for over correction. As shown in simulations by \citet{miller2012reconfiguration}, the presence of a flexible tether on the leaves, such as their stems, results in increased vortex shedding, causing much more erratic fluttering and increased drag forces. For this reason, plants are left to vary in levels of rigidity in their leaf stems in an effort to minimize drag while also avoiding excessive vortex shedding. 
    
One specimen of interest in this struggle is \Citrusxparadisi\ (grapefruit), whose winged petioles exhibit a jointed structure. It is possible that such a structure presents a balance between the counterproductive rigidity found by \citet{miller2012reconfiguration} as it combines the rigid nature of the petiole itself with the flexible nature of a joint. The purpose of this study is to investigate the effectiveness of this petiole structure when compared to a structure lacking in all flexibility.

 %\section{Introduction}
\section{Methods and materials}
\begin{itemize}
\item 1 six-year old grapefruit plant
\item 1 cord of known spring constant
\item 1 measuring tape
\item 1 Kaz Inc. Ht-908 15 inch Honeywell Turbo Force Room Air Circulator Fan
\item colored construction paper
\item 1 pair of scissors
\item clear tape
\item 1 Samsung Galaxy S-8 smartphone camera
\item 1 wooden chair
\item flat, rectangular surfaces
\item MATLAB software
\item pipe cleaners
\item nylon rope
\end{itemize}

The first step in the process was to find a cord of known spring constant. In this experiment, a cord from a party mask was found to have a spring coefficient of 0.589 lbf/in. The cord was attached to a nylon rope and tied to pipe cleaners at both ends so that it could be connected to plant stems. The pipe cleaners were included in the spring constant calculation.

The experimental setup for measuring the spring's deflection due to drag is shown in \fref{fig:methods1}. The cord was strung between a heavy chair and the trunk of a grapefruit plant so that it was not slack, and its length was measured using a tape measure. Then, a box fan was placed one foot away from the trunk of the plant. The plant and fan were raised using flat, rectangular objects, such as textbooks, in order to position the plant's canopy in front of the fan. The cord's deflection was measured five times at each of the fan's three speed settings and a slow motion video of the plant's leaves was taken at the highest fan setting.

Next, the cross sectional area of the plant was calculated using color blob detection in Matlab. To do this, black construction paper was cut to the size and shape of the fan's opening and affixed to it. Because air was blowing from relatively close to the plant, only the area which was directly in front of the fan opening would be exposed to wind. The plant was of a complex shape, so it was more efficient to calculate the area of the fan which was not covered by the plant. The setup was photographed from afar and scaled to eliminate the colorful surroundings, as shown in \fref{fig:methods2}. A picture of the fan-shaped paper was also taken against a blue background from the same distance with no citrus obscuring it and scaled by the same amount, which can also be seen in \fref{fig:methods2}. Using Matlab's colorThresholder system, the ratio of covered to uncovered area was found and multiplied by the known paper area to find the area of the grapefruit leaves.

After analyzing the results of the grapefruit plant, a rigid model was created to test the same conditions without the presence of bending. A small leaf was placed against an aluminum Chipotle bowl lid to serve as a stencil, which was cut out using scissors. Then, this model leaf was taped to a wooden pencil so that the pencil would prevent it from flexing, while any tape edges were cut off. The finished leaf model can be seen in 
\fref{fig:methods3}.

The leaf model was tested in the same manner as the grapefruit plant at the highest fan setting. Because the deflection in the cord was quite small, the displacement was measured using slow motion video. To do this, a tape measure with millimeters marked was taped below the cord and the video was taken from directly above. The video device was planted on a solid surface to minimize movement, and the displacement was found by zooming in on the video and pausing when the cord was fully extended and relaxed. 

In order to compare the model area to the area of the grapefruit, a sheet of orange construction paper was photographed with and without the model leaf against it. Then, using the color blob detection technique in Matlab, the ratio of the model area to the paper area was calculated and multiplied by the actual paper area, in square feet. The images for this analysis can be seen in \fref{fig:methods4}.

With the approximate cross sectional areas of the actual grapefruit leaves and the model found, the discussion over how these areas and their corresponding drags differed could begin.


% Fig 1 is good but cluttered, can we zoom in on only the business part; or turn it into a drawing. Altnatively, add callouts and scale bar? Is there a shot more from the side, that shows the experimental rig without foreshortening? 
\begin{figure}
\begin{center}
\includegraphics[width=0.5\columnwidth]{figures/Grapefruit_Setup.jpg} 
\end{center}
\caption{Experimental setup to measure the deflection of a grapefruit plant when exposed to wind.}
\label{fig:methods1}
\end{figure}

% Not sure what you get from Fig 2; I might move it to appendix. 
\begin{figure}
\begin{center}
\includegraphics[width=0.33\columnwidth]{figures/Fan.jpg}
\includegraphics[width=0.33\columnwidth]{figures/Fan1.jpg}
\end{center}
\caption{The exposed area of the fan-shaped construction paper with and without the plant in front of it.}
\label{fig:methods2}
\end{figure}

% Fig 3 is good but zoom in on the two, add callouts and a scalebar, maybe rotate leafs into their normal positions. 
\begin{figure}
\begin{center}
\includegraphics[width=0.5\columnwidth]{figures/Metal_Leaf_Compared.jpg}
\end{center}
\caption{The aluminum grapefruit leaf model (right) with a small jackfruit leaf, for scale.}
\label{fig:methods3}
\end{figure}

% Not sure what Fig 4 adds, might move to appendix. 
\begin{figure}
\begin{center}
\includegraphics[width=0.33\columnwidth]{figures/MetalLeaf.jpg}
\includegraphics[width=0.33\columnwidth]{figures/Paper.jpg}
\end{center}
\caption{The orange construction paper of known area photographed with and without the model leaf overlaid.}
\label{fig:methods4}
\end{figure}

\subsection{Specimens and physical models}
I used a single \SI{6}{y} old specimen of \emph{Citrus x paradisi} (grapefruit) for all measurements. (words about how tall the specimen was and what conditions it was raised in, since plants are plastic). 

To remove the effects of flexibility, I also created a physical model of a single \emph{C. x paradisi} leaf using (x thick) aluminum sheeting from a food container (Chipotle; (location)). To prepare the physical model, I traced an actual leaf and cut the profile of the model to match. The physical model was mounted on a wooden pencil to provide a rigid attachment point compared to the typical flexible leaf petioles on \emph{C. x paradisi}. 

\subsection{Drag measurements}
(fill in this from other stuff and move other stuff to Appendix)


%\subsection{Statistical analyses}
Statistical analyses of the effects of both leaf and fan speed on drag and drag/area were performed using R \citep{r2020} using two-way analysis of variance (ANOVA); plots were prepared using the \lstinline{tidyverse} and \lstinline{ggplot2} libraries \citep{wickham2019tidyverse}. The average drag per unit area was found for the model leaf and the grapefruit leaves.

 %\section{Methods and materials}
\section{Results}
\label{sec:results}

\subsection{Area measurements}
Area measurements obtained using color blob detection/segmentation in \Matlab\ are shown in \fref{tab:results:area}.
\begin{table}
\caption{Area normal to flow for \Cxparadisi\ specimen and rigid physical model of a single grapefruit leaf.}
\label{tab:results:area}
\begin{center}
\begin{tabular}{cc}
\toprule
& area, \si{\meter\squared} \\
\midrule
\Cxparadisi\ grapefruit specimen & 0.223 \\
rigid physical model of leaf (metal) & 0.0346 \\
\bottomrule
\end{tabular}
\end{center}
\end{table}






\subsection{Drag measurements}
\Fref{tab:results:displacement} gives the measured raw displacement data for the \Cxparadisi\ specimen and the rigid physical model of a single grapefruit leaf. The resulting drag estimates are summarized in \fref{tab:results:drag} and figures~\ref{fig:results:drag} and \ref{fig:results:dragarea}. \Fref{fig:results:drag} shows... (what does it show). When the drag is normalized by area, the effect of leaf flexibility is apparent. \Fref{fig:results:dragarea} shows... (what does it show). 

\begin{table}
\caption{Measured raw displacement (\si{\meter}) for \Cxparadisi\ grapefruit specimen and rigid physical model of a single leaf (metal).}
\label{tab:results:displacement}
\begin{center}
\begin{tabular}{cccc}
\toprule
grapefruit & grapefruit & grapefruit & metal leaf \\
speed 1 & speed 2 & speed 3 & speed 3 \\ 
\midrule
%0.156 & 0.250 & 0.313 & 0.200 \\ % inches / cm 
%0.188 & 0.188 & 0.250 & 0.0500 \\
%0.250 & 0.250 & 0.313 & 0.100 \\
%0.188 & 0.219 & 0.313 & 0.200 \\
%0.188 & 0.250 & 0.313 & 0.150 \\
0.00397 & 0.00635 & 0.00794 & 0.00200 \\ % convert all to SI units
0.00476 & 0.00476 & 0.00635 & 0.00050 \\
0.00635 & 0.00635 & 0.00794 & 0.00100 \\
0.00476 & 0.00556 & 0.00794 & 0.00200 \\
0.00476 & 0.00635 & 0.00794 & 0.00150 \\
\bottomrule
\end{tabular}
\end{center}
\end{table}

\begin{table}
\caption{Summary of drag estimates for \Cxparadisi\ grapefruit specimen and rigid physical model of a single leaf (metal).}
\label{tab:results:drag}
\begin{center}
\begin{tabular}{lcccc}
\toprule
& grapefruit & grapefruit & grapefruit & metal leaf \\
& speed 1 & speed 2 & speed 3 & speed 3 \\
\midrule
drag, \si{\newton} & \num{0.051\pm0.008} & \num{0.061\pm0.007} & \num{0.079\pm0.007} & \num{0.014\pm0.007} \\
drag/area, \si{\newton\per\meter\squared} & \num{2.4\pm0.4} & \num{2.9\pm0.3} & \num{3.7\pm0.3} & \num{4.5\pm2.1} \\
\bottomrule
\end{tabular}
\end{center}
\end{table}

\begin{figure}
\begin{center}
\includegraphics{data/results1.pdf}
\end{center}
\caption{Drag (mean$\pm$sd) for \Cxparadisi\ grapefruit specimen (green) and metal rigid physical model (gray) at different fan speeds. The rigid physical model leaf has less drag than the intact \Cxparadisi\ specimen because of smaller area (two-way ANOVA, $p<\num{4.6e-10}$, $n=5$).}
\label{fig:results:drag}
\end{figure}

\begin{figure}
\begin{center}
\includegraphics{data/results2.pdf}
\end{center}
\caption{Drag, normalized by area, (mean$\pm$sd) for \Cxparadisi\ grapefruit specimen (green) and metal rigid physical model (gray) leaves at different fan speeds. Normalized by area, the rigid metal leaf 19\% more drag than the flexible grapefruit leaves at the same speed, however, the differences are not statistically significant consider the small number of replicates and measurement noise (two-way ANOVA, $p=0.3207$, $n=5$). Differences in drag on the \Cxparadisi\ specimen are apparent at different speeds ($p=0.025$).}
\label{fig:results:dragarea}
\end{figure}






\subsection{Leaf movement during flow}
% This bit needs some help. 
\Fref{fig:results:leafmovement} shows frames from the slow motion video of \Cxparadisi\ at fan speed 3. Following the leaves indicated by the number 1, the leaves which were struck head-on by the wind flexed quite far at their joints, exposing roughly half of their surface area to the camera. The leaf indicated with the number 2 in \fref{fig:results:leafmovement} shows the continued vortex shedding on the leaves when wind was blown across them. The leaf pivoted at its joint, but only exposed about half of its area to the wind before returning to its previous state.
\begin{figure}
\begin{center}
\includegraphics[width=0.49\columnwidth]{figures/Snapshot1.jpg}
\includegraphics[width=0.49\columnwidth]{figures/Snapshot2.jpg}\\
\includegraphics[width=0.49\columnwidth]{figures/Snapshot3.jpg}
\includegraphics[width=0.49\columnwidth]{figures/Snapshot4.jpg}
\end{center}
\caption{Example frames from slow motion video of \Cxparadisi\ specimen at fan speed 3. Leaves marked 1 and 2 rotate at the petiole, reducing by half their area normal to the flow.}
\label{fig:results:leafmovement}
\end{figure}

 %\section{Results}
\section{Discussion}
\label{sec:discussion}

\subsection{Leaf flexibility reduces drag relative to a rigid physical model}
Drag/area ratio is 19\% higher in the rigid physical model compared to the live \Cxparadisi\ specimen (\fref{fig:results:dragarea}; \fref{tab:results:drag}). While the small number of replicates and measurement noise limit the strength of my conclusions, the result suggests leaf flexibility is able to reduce drag experienced by the whole plant. 

Slow motion video frames (\fref{fig:results:leafmovement}) suggest a mechanism: bending at the leaf petiole. As \fref{fig:results:leafmovement} shows, the leaves were able to bend quite far at their joints before further deforming to reduce drag. They also avoided rapid flapping through the rigidity offered by the petioles. It would appear that the petioles gave the leaves a pivot point which allowed them to reposition effectively, but the petioles' firm, single degree-of-freedom, connections to the stem kept the leaves from experiencing vortex-induced rotational modes or the complex fluttering observed by \citet{miller2012reconfiguration}.

\subsection{Where is flexibility most useful?}
My observations suggest distal bending at, at the petioles and nearest the leaves, allow for protective load shedding of wind loads in \Cxparadisi. The drag measurements were taken from a point halfway down the trunk of the plant and the model, and therefore did not capture the torque experienced by the petiole joints. Judging from the immense range of motion in all three axes shown in the slow motion videos, it is reasonable to assume that the leaves' movements produced disproportionately large torque on their petiole joints, while producing less drag on the entire plant when compared to a rigid model. 

The petiole joints are more likely to give under stress, before the plant itself is uprooted or a branch detached. Anecdotally, as the plant's caretaker, I have observed that high winds in the spring resulted in loss of several large leaves on this specimen. In each case the petiole was consistently the fracture point. The joints thus play a dual function, reducing the plant's overall drag while also reducing the negative effects of fluttering to the leaves themselves rather than the plant as a whole. In the highest winds, the petioles provide a sacrificial mechanism to reduce load and avoid catastrophic loss of the entire plant. 
 %\section{Discussion}

\section{Acknowledgements}
I thank 4/C Trombetta, 2/C Pak, and 2/C Figueroa for their assistance as classmates in EW282D / EW496. 

% References
\bibliography{smith.bib}

\clearpage
\appendix
\renewcommand{\figurename}{Supplementary Figure}
\renewcommand{\thefigure}{S\arabic{figure}}
\section{Detailed image processing methods}
\label{sec:A}

Four red-green-blue (RGB) color images (\fref{fig:app:images}) were taken using a Samsung S8 smartphone camera: (1) the orange construction paper with the metal rigid physical model of a single leaf; (2) the orange construction paper alone; (3) the fan aperture alone with black construction paper mask on a blue background; and (4) the fan aperture obstructed by the live \Cxparadisi\ grapefruit specimen. Black and orange construction paper were chosen to maximize contrast and allow easy automatic color segmentation. 
\begin{figure}
\begin{center}
%\includegraphics[width=0.49\columnwidth]{figures/MetalLeaf.jpg}
%\includegraphics[width=0.49\columnwidth]{figures/Paper.jpg} \\
%\includegraphics[width=0.49\columnwidth]{figures/Fan1.jpg}
%\includegraphics[width=0.49\columnwidth]{figures/Fan.jpg}
\includegraphics{figures/fig7.png}
\end{center}
\caption{(A) Metal rigid physical model of a leaf on orange background; (B) orange background; (C) fan aperture; (D) fan aperture obstructed by the live \Cxparadisi\ specimen.}
\label{fig:app:images}
\end{figure}


\subsection{Color blob detection/segmentation using \Matlab}
Color blob detection/segmentation was obtained using the \lstinline{colorThresholder} tool in \Matlab. Images were first converted from RGB colorspace to hue-saturation-value (HSV) colorspace; thresholding was then applied in each channel using the sliders to obtain a binary image corresponding to (1) the area of the orange sheet obstructed by the metal leaf; (2) the orange sheet; (3) the fan; (4) the area of the fan obstructed by the plant, respectively. 

\subsection{Area calculation}
The binary images were labeled as contiguous regions using the \lstinline{bwlabel} command; the largest blobs were manually selected for futher processing. From the labeled region, the \lstinline{regionprops} command was used to obtain the \lstinline{FilledArea} property, which corresponds to the number of pixels contained in the selected blob. 

Using known dimensions of the paper, the area of leaves could be found by calculating the ratio of exposed to unexposed paper area. In the case of the orange construction paper, the paper physical dimensions were \SI{9x12}{\inch} (\SI{0.229x0.305}{\meter}). In the case of the fan, the disk area was \SI{8.5}{\inch} (\SI{0.216}{\meter}) in diameter; these provided the scaling between pixel area and physical area:
\begin{lstlisting}
paperarea=9*12/144;
blackarea=8.5^2*pi/(4*144);
metalarea=leafsize.FilledArea/papersize.FilledArea*paperarea;
grapefruitarea=blackarea-blackarea*fansize/fan1size.FilledArea;
\end{lstlisting}

Area analysis code is provided in the file \lstinline{data/matlab/Calculating Areas.m} in the Github repository at:\\ \url{https://github.com/ew282d-evangelista/ew282d-sp2020-0451-smith}. 




\end{document}
