\section{Introduction}
%Smith: The main ones to look at closely are Vogel and also Laura Miller and Company, and then the algae stuff from Emily Carrington, Mark Denny, Hannah Stewart, etc. May also search on ecological safety factors, reconfigurable structures, use of physical models. I didn't check mangrove literature for drag and shape reconfiguration, I think there's also coral literature about corals breaking during hurricanes to reduce loads / save the colony from death. In engineering context, this is same idea as protective load shedding in electrical systems, or crumple zones and holding bulkheads in structural systems; reefing during sailing, etc. 

%Work from an inverted triangle (broader topics to more specific). Explain what you are interested, review some relevant literature, and then set up what your specific research question is. Cite literature using the author-year format, as in \citep{buck2020go}. If you need pictures to explain the relevant biomechanics, feel free to include. This section should also say a little why your research matters. 
% things to explain include drag, how it depends on speed, area, etc; why it matters to plants, what they can do to minimize or mitigate drag...
%The last part of this section should be the specific hypotheses you seek to test. 

Plants possess unique structural designs and material properties which allow them to combat the negative effects of drag in their environments. As explained by \citep{delangre,2008}, fluid flow over plant canopies creates a drag force on the structure through both pressure difference and friction. In the classic drag equation, both of these factors are combined into the singular drag coefficient, creating the relationship
\[F=0.5*CpV^2A\]where \textit{F} is the total drag force, \textit{C} is the drag coefficient, \textit{p} is the fluid density, \textit{V} is relative fluid velocity, and \textit{A} is the cross-sectional area of the surface normal to the flow direction. As found by \citep{vogel,1989}, plants use their flexible structures to reconfigure under drag load, rolling their leaves into cones in order to minimize cross sectional area, drag coefficient, and the oscillations of vortex shedding. This allows plants to endure the onslaught of heavy winds or floodwaters with much lower risk for structural damage than if they were to be entirely rigid. 
	However, there is room for over correction. As shown in simulations by \citep{miller, et al,2012}, the presence of a flexible tether on the leaves, such as their stems, results in increased vortex shedding, causing much more erratic fluttering and increased drag forces. For this reason, plants are left to vary in levels of rigidity in their leaf stems in an effort to minimize drag while also avoiding excessive vortex shedding. 
    One specimen of interest in this struggle is \emph{Citrus x paradisi} (grapefruit), whose winged petioles exhibit a jointed structure. It is possible that such a structure presents a balance between the counterproductive rigidity found by Miller, et al, as it combines the rigid nature of the petiole itself with the flexible nature of a joint. The purpose of this study is to investigate the effectiveness of this petiole structure when compared to a structure lacking in all flexibility.

