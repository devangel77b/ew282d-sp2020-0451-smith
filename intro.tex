\section{Introduction}
%Smith: The main ones to look at closely are Vogel and also Laura Miller and Company, and then the algae stuff from Emily Carrington, Mark Denny, Hannah Stewart, etc. May also search on ecological safety factors, reconfigurable structures, use of physical models. I didn't check mangrove literature for drag and shape reconfiguration, I think there's also coral literature about corals breaking during hurricanes to reduce loads / save the colony from death. In engineering context, this is same idea as protective load shedding in electrical systems, or crumple zones and holding bulkheads in structural systems; reefing during sailing, etc. 

Work from an inverted triangle (broader topics to more specific). Explain what you are interested, review some relevant literature, and then set up what your specific research question is. Cite literature using the author-year format, as in \citep{buck2020go}. If you need pictures to explain the relevant biomechanics, feel free to include. This section should also say a little why your research matters. 
% things to explain include drag, how it depends on speed, area, etc; why it matters to plants, what they can do to minimize or mitigate drag...
The last part of this section should be the specific hypotheses you seek to test. 

