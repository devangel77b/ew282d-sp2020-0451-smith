\section{Introduction}
Many organisms are subject to environmental flows and the forces that result from flow. Freely moving organisms may have to locomote in high wind or high currents; while sessile organisms may need to remain attached and undamaged during normal conditions as well as during extreme events \citep{vogel1994life, vogel2003comparative}. For example, intertidal algae must remain attached to the bottom with a holdfast and must have stipes with sufficient mechanical strength for a variety of hydrodynamic loading conditions from drag and added mass \citep{carrington1992consequences, denny2002mechanics, stewart2004hydrodynamic, stewart2006hydrodynamic, boller2007interspecific}. Mechanistic models of forces may be combined with statistics of extreme events and measures of environmental safety factors to determine how long an organism or a community might be expected to persist in a particular environment \citep{denny2009on}. 

In the case of land plants, fluid forces from wind have important consequences for mechanical design \citep{delangre2008effects} as well as dispersal of seeds and pollen \citep{vogel2003comparative, evangelista2011explosive, stevenson2015when}. Wind can result in both lift and drag on a structure such as a plant; lift is transverse to the flow direction while drag is parallel with it \citep{vogel1994life, vogel2003comparative, kundu2012fluid}. Fluid flow over individual plants or within canopies creates drag on structures through pressure differences due to shape as well as friction in a thin boundary layer nearest the plant \citep{delangre2008effects, shapiro1961shape, kundu2012fluid, vogel1994life, vogel2003comparative}. If the drag exceeds some mechanical limit, the plant may experience mechanical failure by breaking or losing its attachment to the ground. 

Drag can be expressed as \citep{kundu2012fluid}:
\begin{equation}
D=0.5 C_D \rho V^2 A,
\label{eq:drag}
\end{equation}
where $D$ is the total drag force, $C_D$ is a nondimensional drag coefficient that is dependent on shape and on Reynolds number ($\operatorname{Re}$, a nondimensional ratio of inertial and viscous forces), $\rho=\SI{1.204}{\kilo\gram\per\meter\cubed}$ is the density of air, $V$ is relative fluid velocity, and $A$ is a reference area, here taken as the cross-sectional area of the surface normal to the flow direction. From \fref{eq:drag}, it is clear that $D$ can be minimized by reducing $C_D$ via changes in shape to a more streamlined form; reduction of $A$ via feathering into the wind, bending or curling of leaves \citep{ennos2000functional}, or sacrificial loss of branches to save the tree; and reduction of $V$ by bending/hunkering down in the boundary layer nearest the ground. Thus, plants can use their flexible structures to reconfigure under drag load, rolling their leaves into cones in order to minimize cross sectional area, drag coefficient, and the oscillations of vortex shedding \citep{miller2012reconfiguration, vogel1989drag, ennos2000functional}. This allows plants to endure the onslaught of heavy winds or floodwaters with much lower risk for structural damage than if they were to be entirely rigid. From an engineering point of view, these techniques are similar to feathering a propeller, reefing a sail, or protective load shedding in electrical systems.

In some cases, the interplay of elastic behavior of a flexible stem and the fluid mechanics as shape changes can result in dynamic flutter and other similar behaviors \citep{miller2012reconfiguration, boller2007interspecific, denny2002mechanics}. For the case of land plants, as shown in simulations by \citet{miller2012reconfiguration}, the presence of a flexible tether on the leaves, such as their stems, results in increased vortex shedding, causing much more erratic fluttering and increased drag forces. For this reason, plants vary in levels of rigidity in their leaf stems, in an effort to minimize drag while also avoiding excessive vortex shedding \citep{miller2012reconfiguration, vogel2009leaves}. 
    
Issues of plant response to high wind loads resulting in high drag are especially interesting in the case of tropical fruits, which are economically important and also grow natively in areas subject to hurricanes and cyclones. \citet{ennos2000functional} considered the mechanical role of the leaf petiole (the stem portion attaching at the proximate end of the leaf) in bananas (\emph{Musa textilis}) and found its U-shaped cross section allowed it sufficient flexural rigidity to hold up the broad leaves during normal conditions, but also provided sufficient torsional compliance to allow the leaves to feather into the wind during high loads. 
    
Another economically important tropical fruit with an interesting leaf petiole is the grapefruit, \Citrusxparadisi\ (Macfad.), whose winged petioles (see \fref{fig:methods:specimens}B) exhibit a peculiar jointed structure \citep{morton1987grapefruit, kumamoto1987mystery, macfayden1837flora}. Winged petioles are found in \emph{Citrus} fossils from both the Pliocene of Italy \citep{fischer1998citrus} and Miocene of Yunan, China \citep{xie2013citrus} and may be synapomorphic for the genus. It is possible that such a structure presents a balance between the counterproductive rigidity found by \citet{miller2012reconfiguration} as it combines the rigid nature of the petiole itself with the flexible nature of a joint. It may also constrain bending to certain preferential directions depending on the magnitude and direction of the wind loading on an individual leaf. The purpose of this study is to investigate the effectiveness of this petiole structure when compared to a structure lacking in all flexibility. I hypothesize that, compared to a rigid leaf, a live \Cxparadisi\ plant will bend and reconfigure its shape as flow increases, achieving lower drag either through static bending or through time-varying flutter processes that provide lower average force overall than the rigid condition. Due to the current COVID-19 outbreak, I will test my hypotheses by measuring forces with an elastic flexure-based apparatus \citep{denny1983simple, bell1984quantifying} on a live specimen compared with a rigid physical model \citep{stevenson2015when, evangelista2014shifts, stewart2006hydrodynamic, vogel2009leaves}. 

